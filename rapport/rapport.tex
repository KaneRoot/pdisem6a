%%%%%%%%%%%%%%%%%%%%%%%%%%%%%%%%%%%%%%%%%
% Journal Article
% LaTeX Template
% Version 1.0 (25/8/12)
%
% This template has been downloaded from:
% http://www.LaTeXTemplates.com
%
% Original author:
% Frits Wenneker (http://www.howtotex.com)
%
% License:
% CC BY-NC-SA 3.0 (http://creativecommons.org/licenses/by-nc-sa/3.0/)
%
%%%%%%%%%%%%%%%%%%%%%%%%%%%%%%%%%%%%%%%%%

%----------------------------------------------------------------------------------------
%	PACKAGES AND OTHER DOCUMENT CONFIGURATIONS
%----------------------------------------------------------------------------------------

%\documentclass[12pt,a4paper,utf8x]{report}
\documentclass[twoside]{article}

% rajout personnel
\usepackage{graphicx}
\usepackage [frenchb]{babel}
\usepackage{ucs}
\usepackage[utf8x]{inputenc}

\usepackage[sc]{mathpazo} % Use the Palatino font
\usepackage[T1]{fontenc} % Use 8-bit encoding that has 256 glyphs
\linespread{1.05} % Line spacing - Palatino needs more space between lines
\usepackage{microtype} % Slightly tweak font spacing for aesthetics

\usepackage[hmarginratio=1:1,top=32mm,columnsep=20pt]{geometry} % Document margins
\usepackage{multicol} % Used for the two-column layout of the document
\usepackage[colorlinks=true, urlcolor=cyan, pdftitle={Projet PDI}]{hyperref}

\usepackage[hang, small,labelfont=bf,up,textfont=it,up]{caption} % Custom captions under/above floats in tables or figures
\usepackage{booktabs} % Horizontal rules in tables
\usepackage{float} % Required for tables and figures in the multi-column environment - they need to be placed in specific locations with the [H] (e.g. \begin{table}[H])

\usepackage{lettrine} % The lettrine is the first enlarged letter at the beginning of the text
\usepackage{paralist} % Used for the compactitem environment which makes bullet points with less space between them

\usepackage{abstract} % Allows abstract customization
\renewcommand{\abstractnamefont}{\normalfont\bfseries} % Set the "Abstract" text to bold
\renewcommand{\abstracttextfont}{\normalfont\small\itshape} % Set the abstract itself to small italic text

\usepackage{titlesec} % Allows customization of titles
\titleformat{\section}[block]{\large\scshape\centering{\Roman{section}.}}{}{1em}{} % Change the look of the section titles 

\usepackage{fancyhdr} % Headers and footers
\pagestyle{fancy} % All pages have headers and footers
\fancyhead{} % Blank out the default header
\fancyfoot{} % Blank out the default footer
\fancyhead[C]{Programmation Distribuée $\bullet$ décembre 2012} % Custom header text
\fancyfoot[RO,LE]{\thepage} % Custom footer text

%----------------------------------------------------------------------------------------
%	TITLE SECTION
%----------------------------------------------------------------------------------------

\title{\vspace{-15mm}\fontsize{24pt}{10pt}\selectfont\textbf{Programmation Distribuée}}

\author{
\large
\textsc{Philippe PITTOLI - Claude HEMBERGER}\\[2mm]
\normalsize Université de Strasbourg \\ 
\normalsize \href{mailto:philippe.pittoli@etu.unistra.fr}{nous contacter}
\vspace{-5mm}
}
\date{}

%----------------------------------------------------------------------------------------

\begin{document}

\maketitle % Insert title

\thispagestyle{fancy} % All pages have headers and footers

%----------------------------------------------------------------------------------------
%	ARTICLE CONTENTS
%----------------------------------------------------------------------------------------

%\begin{multicols}{1} % Two-column layout throughout the main article text

\section{Introduction}
Nous avons estimer que des cellules qui meurent était un sujet trop triste pour ce projet.
C'est pourquoi nous avons décidé de modifier le thème en quelque chose de plus joyeux.
Ainsi, à la place de faire mourir des cellules, nous préférons tuer des petits chatons.

Comme nous avons tout de suite changé de thème, aucune cellule n'a du mourir pour ce projet.

\section{Implémentation du projet}
Nous avons découpé le projet en plusieurs parties distinctes : 
\begin{itemize}
	\item Le serveur 
		\begin{itemize}
			\item il crée un « KittyCluster »\protect\footnote{
					KittyCluster : un ensemble quelconque de blocs de 32*32 cellules (ah non, chatons !).
				} de taille variable\protect\footnote{
					Pour faire varier la taille du KittyCluster principal, 
						 voir le champ « CARNAGE\_FIELD\_SIZE » dans le makefile.
				} ;
			\item partage ce KittyCluster en plusieurs à envoyer aux clients ;
			\item récupère le résultat sous la forme de « KittyHistory » ;
			\item réassemble le tout pour former des snapshots cohérents sur plusieurs tours ;
			\item peut envoyer un snapshot d'un instant T du KittyCluster principal ;
		\end{itemize}
	\item Le client
		\begin{itemize}
			\item va chercher un permis pour tuer des chatons ;
			\item fait des calculs sur le KittyCluster envoyé  
			\item renvoie un historique (KittyHistory) ;
		\end{itemize}
	\item WatchTheKittiesDie : c'est le visualiseur, qui regarde l'évolution du jeu.
\end{itemize}

\section{Effectuer des tests}
Les 3 choses à faire pour lancer le tout (chacun dans un shell distinct) :
\begin{itemize}
	\item make lancer\_serveur
	\item make lancer\_client
	\item make watch\_kitties
\end{itemize}
Et pour kill rmiregistry :
	make destroy
%------------------------------------------------


%\end{multicols}

\end{document}
