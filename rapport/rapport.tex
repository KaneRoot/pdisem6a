%%%%%%%%%%%%%%%%%%%%%%%%%%%%%%%%%%%%%%%%%
% Journal Article
% LaTeX Template
% Version 1.0 (25/8/12)
%
% This template has been downloaded from:
% http://www.LaTeXTemplates.com
%
% Original author:
% Frits Wenneker (http://www.howtotex.com)
%
% License:
% CC BY-NC-SA 3.0 (http://creativecommons.org/licenses/by-nc-sa/3.0/)
%
%%%%%%%%%%%%%%%%%%%%%%%%%%%%%%%%%%%%%%%%%

%----------------------------------------------------------------------------------------
%	PACKAGES AND OTHER DOCUMENT CONFIGURATIONS
%----------------------------------------------------------------------------------------

%\documentclass[12pt,a4paper,utf8x]{report}
\documentclass[twoside]{article}

% rajout personnel
\usepackage{graphicx}
\usepackage [frenchb]{babel}
\usepackage{ucs}
\usepackage[utf8x]{inputenc}

\usepackage[sc]{mathpazo} % Use the Palatino font
\usepackage[T1]{fontenc} % Use 8-bit encoding that has 256 glyphs
\linespread{1.05} % Line spacing - Palatino needs more space between lines
\usepackage{microtype} % Slightly tweak font spacing for aesthetics

\usepackage[hmarginratio=1:1,top=32mm,columnsep=20pt]{geometry} % Document margins
\usepackage{multicol} % Used for the two-column layout of the document
\usepackage[colorlinks=true, urlcolor=cyan, pdftitle={Projet PDI}]{hyperref}

\usepackage[hang, small,labelfont=bf,up,textfont=it,up]{caption} % Custom captions under/above floats in tables or figures
\usepackage{booktabs} % Horizontal rules in tables
\usepackage{float} % Required for tables and figures in the multi-column environment - they need to be placed in specific locations with the [H] (e.g. \begin{table}[H])

\usepackage{lettrine} % The lettrine is the first enlarged letter at the beginning of the text
\usepackage{paralist} % Used for the compactitem environment which makes bullet points with less space between them

\usepackage{abstract} % Allows abstract customization
\renewcommand{\abstractnamefont}{\normalfont\bfseries} % Set the "Abstract" text to bold
\renewcommand{\abstracttextfont}{\normalfont\small\itshape} % Set the abstract itself to small italic text

\usepackage{titlesec} % Allows customization of titles
\titleformat{\section}[block]{\large\scshape\centering{\Roman{section}.}}{}{1em}{} % Change the look of the section titles 

\usepackage{fancyhdr} % Headers and footers
\pagestyle{fancy} % All pages have headers and footers
\fancyhead{} % Blank out the default header
\fancyfoot{} % Blank out the default footer
\fancyhead[C]{Programmation Distribuée $\bullet$ décembre 2012} % Custom header text
\fancyfoot[RO,LE]{\thepage} % Custom footer text

%----------------------------------------------------------------------------------------
%	TITLE SECTION
%----------------------------------------------------------------------------------------

\title{\vspace{-15mm}\fontsize{24pt}{10pt}\selectfont\textbf{Programmation Distribuée}}

\author{
\large
\textsc{Philippe PITTOLI - Claude HEMBERGER}\\[2mm]
\normalsize Université de Strasbourg \\ 
\normalsize \href{mailto:philippe.pittoli@etu.unistra.fr}{nous contacter}
\vspace{-5mm}
}
\date{}

%----------------------------------------------------------------------------------------

\begin{document}

\maketitle % Insert title

\thispagestyle{fancy} % All pages have headers and footers

%----------------------------------------------------------------------------------------
%	ARTICLE CONTENTS
%----------------------------------------------------------------------------------------

%\begin{multicols}{1} % Two-column layout throughout the main article text

\section{Introduction}
Nous avons estimé que des cellules qui meurent était un sujet trop triste pour ce projet.
C'est pourquoi nous avons décidé de modifier le thème en quelque chose de plus joyeux et inutile.
Ainsi, à la place de faire mourir des cellules, nous préférons tuer des petits chatons.

Comme nous avons changé de thème en tout début, aucune cellule n'a dû mourir pour ce projet.

\section{Implémentation du projet}
\subsection{Découpage du projet}
Nous avons découpé le projet en plusieurs parties distinctes : 
\begin{itemize}
	\item Le serveur 
		\begin{itemize}
			\item crée un « KittyCluster »\protect\footnote{
					KittyCluster : un ensemble quelconque de blocs de 32*32 cellules (ah non, chatons !).
				} de taille variable\protect\footnote{
					Pour faire varier la taille du KittyCluster principal, 
						 voir le champ « CARNAGE\_FIELD\_SIZE » dans le makefile.
				} ;
			\item partage ce KittyCluster en plusieurs plus petits à envoyer aux clients ;
			\item récupère le résultat sous la forme de « KittyHistory »\protect\footnote{
					KittyHistory : Permet de conserver les résultats de calcul sur X tours.
				}	;
			\item réassemble le tout pour former des snapshots cohérents sur plusieurs tours ;
			\item peut envoyer un snapshot d'un instant T du KittyCluster principal ;
		\end{itemize}
	\item Le client
		\begin{itemize}
			\item va chercher un permis pour tuer des chatons ;
			\item fait des calculs sur le KittyCluster envoyé ;
			\item renvoie un historique (KittyHistory) ;
		\end{itemize}
	\item WatchTheKittiesDie : c'est le visualiseur, qui regarde l'évolution du jeu.
\end{itemize}
\subsection{Le déroulement normal}
Tout d'abord, nous devons lancer le serveur qui se charge de créer la zone d'une certaine taille via des paramètres qu'on lui passe\protect\footnote{
Tout est fait automatiquement via notre makefile.
} puis il se met en attente de clients.
À chaque client qui se connecte, le serveur fait un calcul pour savoir comment il faut découper les tâches à attribuer.

Lorsqu'un client se connecte, il demande un numéro de client (une licence pour tuer dans notre jargon), puis il demande des zones à calculer (un « KittyCluster »).
Le serveur lui envoie donc une zone de taille variable suivant le nombre de clients et la taille initiale de la zone globale.

Une fois le calcul terminé (qui se fait dans la classe « Computation » ), le client renvoie un historique de ce qu'il a calculé (« KittyHistory ») via la méthode « resultsOfTheGenocide ».
De là, le serveur est en charge de réassembler la zone calculée sur $x$ générations avec le reste de la zone en construction.

Le client continue à demander des zones à calculer.

Pour visualiser la zone, nous n'utilisons pas directement le serveur, mais la classe « WatchTheKittiesDie ».

\subsection{Remarques et choix techniques}
Nous utilisons RMI, car nous avions déjà vu ce système lorsque nous avons commencé le développement du projet.
Nous savons ce système gourmand en ressources réseau, d'où notre implémentation nous permettant $x$ générations sans utilisation du réseau.

Notre implémentation permet une utilisation à grande échelle (plusieurs dizaines de clients).
Du fait que les clients calculent 32 tours à l'avance, on limite l'utilisation du réseau qui nous semblait être le point le plus critique.
De ce fait, ajouter des clients n'implique pas une consommation excessive du réseau.
Il n'y a donc pas d'attente réseau ou d'interruption, le calcul de grandes quantités de données peut se faire en une seule fois.

Ceci implique une redondance dans les calculs.
Cependant, plus les kittyclusters à calculer par le client sont grands, et moins la redondance impactera sur la performance globale.
Nous ne trouvons pas cela problématique, car pour profiter et justifier le besoin d'une approche distribuée, on peut supposer le champ de bataille très grand.

Nous n'avons pas joué sur les threads pour faire en sorte qu'un client calcule $x$ parties de zone à afficher.
Il suffit de lancer le client plusieurs fois sur la même machine.
D'ailleurs, nous avons ajouté un script permettant de faire ainsi sur les machines Linux à titre d'exemple.

Nous avons séparé la visualisation de la zone du serveur parce que nous pensons que cela peut représenter une somme de calculs assez conséquente et le serveur a déjà beaucoup de calculs à faire. 
Lui rajouter une couche de Swing, avec une grande zone à afficher, l'aurait rendu moins performant.
L'affichage se fait donc (potentiellement) sur un autre ordinateur.

\section{Effectuer des tests}
Les 3 choses à faire pour lancer le tout (chacun dans un shell distinct, après avoir compilé le projet avec make) :
\begin{description}
	\item make lancer\_serveur
	\item make lancer\_client ou make lancer\_x\_clients
	\item make watch\_kitties
\end{description}
Et pour kill rmiregistry :
	make destroy

\section{Outils utilisés}
Dans notre quête envers la complétion du projet ainsi que l'extermination de la race féline, nous avons fait usage de divers armes et équipements:
\begin{itemize}
    \item Vim : à la manière de la potion magique d'Astérix, ce breuvage a démultiplié notre productivité, et nous a permis d'éditer et corriger notre code source en un tour de main.
    \item Git : compléter une quête en binôme n'a jamais été facile. 
	Surtout dans le cas d'une relation à distance, comme ce fut le cas dans ce projet. 
	Mais à l'aide de git, et de sa nature décentralisée, chacun de nous a pu travailler dans son coin, tout en travaillant ensemble.
    \item Java : cette arme est bel et bien une épée à double tranchant. 
	Elle est effectivement utilisable par n'importe qui, sur n'importe quelle plateforme. 
	Cependant, sa conception lourde et compliquée rend cette épée fatigante et frustrante à utiliser.
    \item \LaTeX : après en avoir fini avec notre génocide, nous avons décidé d'engager ce ménestrel pour compter nos fabuleuses péripéties et douteuses décisions.
        Ses qualités principales sont sa propreté et son professionnalisme, qualités rares de nos jours.
\end{itemize}

\section{Répartition du travail}
Concernant la partie serveur, client, visualisation, calcul du jeu de la vie, makefile : Philippe PITTOLI.
Pour le partage des tâches entre clients, la partie RMI et la gestion de déconnexion de clients : Claude HEMBERGER.

Nous avons fait ensemble la gestion d'un « bloc » qui représente 32*32 cellules (la partie avec des calculs binaires sur des entiers).

Globalement, nous avons travaillé tous les deux sur chaque partie du projet (relecture du code, corrections diverses, apport thématique aux noms des classes et variables).

\begin{center} 
	\includegraphics[scale=0.5]{img/chaton.jpg}
\end{center}

\end{document}
